\setlength{\parindent}{2em}ImageNet数据集包含大概22000种共1500多万的带标签的高清图片。这些图片是从网络上搜集的,由亚马逊的Mechanical  Turkey工具进行人工标记。作为PASCAL视觉目标挑战赛的一部分,一年一度的ImageNet大型视觉识别挑战赛(ILSVRC)从2010年开始就已经在举办了。ILSVRC使用ImageNet的一个子集,这个子集大概包含1000个类别,每个类别大概包含1000张图片。总共大概有120万张训练图像,5万张验证图像和15万张测试图像。\\

\setlength{\parindent}{2em}ILSVRC-2010是ILSVRC中唯一能获得测试集标签的版本,因此这也就是我们完成大部分实验的版本。由于我们同样在ILSVRC-2012上输入了模型,所以我们在第六节中也讨论了这个数据集上的结果,可是这个测试集标签无法获得。在ImageNet上,通常检验两类错误率:top-1和top-5,其中top-5误差率是指测试图像上正确标签不属于被模型认为是最有可能的五个标签的百分比。\\

\setlength{\parindent}{2em}ImageNet由各种分辨率的图像组成,而我们的系统需要一个恒定的输入维数。因此,我们对图片进行采样,获得固定大小的256X256的分辨率。给定一张矩形图像,我们首先重新缩放图像,使得短边长度为256,然后取中心区域的256X256像素。除了将每个像素中减去训练集的像素均值之外,我们没有以任何其他方式对图像进行预处理。所以我们在像素的(中心)原始RGB值上训练了我们的网络。\\

