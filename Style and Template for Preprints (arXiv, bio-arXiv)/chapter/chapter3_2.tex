单个的GTX580 GPU只有3GB内存,这限制了可以在其上训练的网络的最大规模。事实证明,120万个训练样本足以训练那些因规模太大而不适合使用一个GPU训练的网络。因此,我们将网络分布在两个GPU上。目前的GPU很适合于跨GPU并行化操作,因为它们能够直接读写对方的内存,而无需通过主机内存。我们采用的这种并行模式主要是将一半的网络内核(或神经元)放在每个GPU上,然后再采用一个小技巧:将GPU通信限制在某些特定的层上。这意味着,比如,第三层的内核从所有的第二层内核映射(kernel map)中获得输入,但是,第四层的内核只从和自己在同一个GPU上的第三层内核中获得输入。选择连接模式对于交叉验证是一个不小的问题,但这使得我们能够精确调整通信量,直到它的计算量的达到可接受的程度。\\

由此产生的结构有点类似于Ciresan等人提出的“柱状”CNN的结构[5],不同之处在于我们的纵列不是独立的(见图2)。与一个GPU上训练的每个卷积层只有一半的内核数量的网络相比,该方案分别将我们的top-1和top-5错误率分别降低了1.7%和1.2%。双GPU结构网络比单GPU网络$\footnote{实际上单GPU网络与双GPU网络在最后的卷积层有着相同数量的核。这是因为大多数网络的参数在第一个全连接层,这需要上一个卷积层作为输入。所以,为了使两个网络有数目大致相同的参数,我们不把最后一个卷积层大小减半(也不把它后面跟随的全连接层减半)。因此,这种比较关系更偏向有利于单GPU网络,因为它比双GPU网络的“一半大小”要大}$所需的训练时间要稍微少一些。\\
	