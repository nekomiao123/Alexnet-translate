CNN中的池化层总结了同一核映射上相邻组神经元的输出。一般来说,相邻池化单元总结的邻近关系是不重叠的(例如[17,11,4])。更确切的说,池化层可看作由池化单元网格组成,网格间距为s个像素,每个网格归纳池化单元中心位置$z\times z$大小的邻居。如果设置$s=z$,我们会得到通常在CNN中采用的传统局部池化。如果设置$s < z$,我们会得到重叠池化。这就是我们网络中使用的方法,我们设置$s = 2$,$z = 3$。这个方案分别降低了top-1  0.4\%和top-5  0.3\%的错误率,与非重叠方案s = 2,z = 2相比,输出的维度是相等的。我们通常观察到在训练过程中有重叠池化的更加难以过拟合一些。