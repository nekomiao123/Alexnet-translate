\setlength{\parindent}{2em}当前的目标识别方法基本上都使用了机器学习的方法。为了提高这些方法的性能,我们可以收集更多的数据,学习得到更加强大的模型,用更好的方法来防止过拟合。直到现在,有标签的数据集都是比较小的,一般只有万张的数量级(如NORB[16],Caltech-101/256[8,9],以及CIFAR-10/100[12])。在这个大小的数据集上,简单的识别任务可以得到很好的解决,尤其是在通过标签保留变换进行数据增强的情况下。例如,目前在MNIST数据集上面数字识别最小的错误率(<0.3\%)已经接近了人类的水平[4]。但是现实世界中的目标呈现出相当大的可变性,所以要去学习识别他们就需要使用更大的训练数据集。实际上,人们也已广泛地认识到小图像数据集的缺点(如Pinto等[21]),但直到最近,收集包含数百万图像的带标签数据集才成为可能。这些新的大型数据集包括LabelMe[23](包含数十万张被完全分割的图片),ImageNet[6](由1500万张被标记的高清图片组成,覆盖了2.2万个类别)\\

\setlength{\parindent}{2em}为了从数百万张图片中学习到数千种目标,我们需要一个学习能力极强的模型。然而,物体识别任务极高的复杂度意味着即使拥有ImageNet这么大的数据集,这个问题也很难被具体化。所以我们的模型也需要大量先验知识去弥补我们缺失的数据。卷积神经网络(CNNs)就是一种这样的模型[16,11,13,18,15,22,26]。他们的学习能力可以通过控制网络的深度和宽度来调整,他们也可以对图像的本质做出强大且基本准确的假设(也就是说,统计上的稳定性,以及像素依赖的局部性)。因此,与相似大小的标准前馈神经网络相比,CNNs的连接和参数更少,所以更易训练,而他们理论上的最佳性能仅比标准前馈神经网络稍差一点。\\

\setlength{\parindent}{2em}尽管CNNs有很棒的质量,和更有效率的局部结构,但将他们大规模的应用到高分辨率的图像中仍然需要高昂的代价。幸运的是,当前的GPU搭配上高度优化的2D卷积实现,已经足够强大到去加速大型CNNs的训练过程,并且最近的数据集例如ImageNet已经包含足够的有标签样本,能够训练出不会严重过拟合的模型。\\

\setlength{\parindent}{2em}本文的具体贡献如下:我们在ImageNet的子集ILSVRC-2010与ILSVRC-2012[2]上训练了目前为止最大的卷积神经网络之一,并且在这个数据集上达到了迄今为止最好的结果。我们编写了高度优化的2D卷积GPU实现,以及其他所有训练卷积神经网络的固有操作,这些都已公开。我们的神经网络包含一系列新的不同凡响的特征,这提高了它的表现,也减少了训练时间,具体情况会在第三节介绍。即使我们拥有120万的标签样本,我们网络巨大的体积也使得过拟合成为了一个严重的问题,所以我们使用了一系列有效的技术去防止过拟合,具体情况会在第四节介绍。我们最终的神经网络包含5个卷积层和3个全连接层,这个深度似乎是很重要的:我们发现去掉任意一个卷积层(每一层包含的参数个数不超过整个模型参数个数的1\%)都会导致更差的表现。\\

\setlength{\parindent}{2em}最后,网络的大小主要受限于目前GPU的内存大小和我们愿意忍受的训练时长。我们的网络在两块GTX 580 3GB的GPU上训练了五六天。我们所有的实验都表明,只要等到更快的GPU和更大的数据集出现,其结果就可以进一步被提高。\\
