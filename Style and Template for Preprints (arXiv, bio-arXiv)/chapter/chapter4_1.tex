减少图像数据过拟合最简单最常用的方法,是使用标签-保留转换,人为地扩大数据集(例如,[25,4,5])。我们使用了两种独特的数据增强方式,这两种方式都可以从原始图像通过非常少的计算量产生变换的图像,因此变换图像不需要存储在硬盘上。在我们的实现中,变换图像通过CPU的Python代码生成,而此时GPU正在训练前一批图像。所以这种放大数据集的方式是很高效很节省计算资源的。\\

第一种数据增强的方式包括图像变换和水平翻转。我们通过从$256\times256$的图像上随机提取$224\times224$(译者注:此处应是227$\times$227)的图像块(及其水平镜像)的方法来实现,并在这些提取的图像$\footnote{这就是在图2中输入为224×224×3(译者注:此处应是227$\times$227$\times$3)的原因.}$上对我们的神经网络进行了训练。这通过一个2048因子增大了我们的训练集,尽管最终的训练样本是高度相关的。如果没有这个方案,我们的网络会出现严重的过拟合,这会迫使我们使用更小的网络。在测试时,网络会抽取五个$224\times224$(译者注:此处应是227$\times$227)的图像块(四个角上的图像块和中心的图像块)以及他们的水平翻转(总共是个图像块)进行预测,然后将softmax层对这十个图像块做出的预测取平均。\\

第二种放大数据集的方法是改变训练图像的RGB通道的强度。具体地,我们在整个ImageNet训练集上对RGB像素进行了PCA(主成分分析)。对于每张训练图像,我们通过均值为0,方差为0.1的高斯分布产生一个随机值a,然后通过向图像中加入更大比例的相应的本征值的a倍,把其主成分翻倍。因此,对于每个RGB像素$I_{xy}=\begin{bmatrix}
I_{xy}^{R} & I_{xy}^{G} & I_{xy}^{B} 
\end{bmatrix}^{T}$,我们加入的值如下:\\
$$
\begin{bmatrix}
P_{1} & P_{2} & P_{3}
\end{bmatrix}
\begin{bmatrix}
\alpha _{1}\lambda _{1} & \alpha_{2}\lambda _{2} & \alpha _{3}\lambda _{3}
\end{bmatrix}^{T}
$$

其中,$P_{i}$和$\lambda _{i}$分别是RGB像素值$3\times3$协方差矩阵的第$i$个特征向量和特征值,$\alpha ^{i}$是前面提到的随机变量。对于某个训练图像的所有像素,每个$\alpha ^{i}$只获取一次,直到图像进行下一次训练时才重新获取。这个方案近似抓住了自然图像的一个重要特性,即光照的颜色和强度发生变化时,目标是不变的。这一方法把top-1错误降低了1\%。
